\documentclass[12pt, twoside]{article}
\usepackage{jmlda}
\newcommand{\hdir}{.}

\begin{document}

\title
    [] % краткое название; не нужно, если полное название влезает в~колонтитул
    {Условия существования петель скрытой обратной связи в рекомендательных системах}
\author
    [А.\,А.~Пилькевич] % список авторов (не более трех) для колонтитула; не нужен, если основной список влезает в колонтитул
    {А.\,А.~Пилькевич, A.\,C.~Хританков} % основной список авторов, выводимый в оглавление
    [А.\,А.~Пилькевич$^1$, A.\,C.~Хританков$^2$] % список авторов, выводимый в заголовок; не нужен, если он не отличается от основного
\email
   {anton39reg@mail.ru; anton.khritankov@gmail.com}
%\thanks
%    {Работа выполнена при
%     %частичной
%     финансовой поддержке РФФИ, проекты \No\ \No 00-00-00000 и 00-00-00001.}
%\organization
%    {$^1$Организация, адрес; $^2$Организация, адрес}
\abstract
  {В работе исследуется эффект  петель скрытой обратной связи в рекомендательных системах.
  Решается задача поиска условий возникновения положительной обратной связи для системы c алгоритмом Thomson Sampling Multi-armed Bandit с учётом наличия шума в выборе пользователя.
  Под положительной обратной связью подразумевается неограниченный рост интереса пользователя к предлагаемым объектам. 
  Без шума известно, что всегда существуют условия неограниченного роста. 
  Экспериментально проверяются полученные условия в имитационной модели.

\bigskip
\noindent
\textbf{Ключевые слова}: \emph {machine learning; hidden feedback loops; echo chamber; filter bubble}
}

%данные поля заполняются редакцией журнала
\doi{}
\receivedRus{}
\receivedEng{}

\maketitle
\linenumbers

\section{Введение}
Рекомендательные системы являются важной составляющей социальных сетей, веб-поиска и других сфер. 
Мы будем рассматривать эффект петель скрытой обратной связи, который подразумевает смещение интереса пользователя из-за его взаимодействия с рекомендательной системой. 
Эффект петель скрытой обратной связи в реальных и модельных задачах во многих публикациях описыается как нежелательное явление. 
Частные и часто рассматриваемые случаи скрытой обратной являются echo chamber и filter bubles.
До сих пор нет какой-либо строгой формализации условий возникновения этих эффектов при условиях приближенных к реальности. 

Целью данной работы является нахождение условий существования петель обратной связи в рекоммендательной системе с алгоритмом Thomson Sampling в условиях зашумлённости выбора пользователя.
Зашумлённость выбора рассматривается, как смещение первоночального интереса к исходному объект или категории.
Предлагается способ отыскание требуемых условий модели исходя из теоретических свойств алгоритма TS путём нахождения рекуретного соотношения для регардов.  
Также рассмаривается вариант нахождение этих условий чисто из экспериментов. 
Наибольший интерес представляет матетическое описание искомых условий с дальнейшим экспериментальным подтверждением полученных соотношений.
Для проверки результатов используется имитационная модель, использующая синтетические данные.  

Уже существует модель этого эффекта в случае отсутствия шума в действиях пользователя, что не реализуется на практике. 
Подобное исследование проводилось в статье [1] на примере различных моделей в задаче многорукого бандита. 
Им удалось показать условия существования неограниченного роста интереса пользователя. 
В работе [2] изучалась схожая постановка задачи и были получены условия возникновения, но рассматривалась линейная модель и градиентный бустинг (GBR). 
Важным отличием нашей работы является факт рассмотрения более сложных и приближенных условий модели, таких как шум в выборе пользователя и другой алгоритм рекомендательной системы.  

\section{Название раздела}
Данный документ демонстрирует оформление статьи,
подаваемой в электронную систему подачи статей \url{http://jmlda.org/papers} для публикации в журнале <<Машинное обучение и анализ данных>>.
Более подробные инструкции по~стилевому файлу \texttt{jmlda.sty} и~использованию издательской системы \LaTeXe\
находятся в~документе \texttt{authors-guide.pdf}.
Работу над статьёй удобно начинать с~правки \TeX-файла данного документа.

Обращаем внимание, что данный документ должен быть сохранен в кодировке~\verb'UTF-8 without BOM'.
Для смены кодировки рекомендуется пользоваться текстовыми редакторами \verb'Sublime Text' или \verb'Notepad++'.

\paragraph{Название параграфа}
Разделы и~параграфы, за исключением списков литературы, нумеруются.

\section{Заключение}
Желательно, чтобы этот раздел был, причём он не~должен дословно повторять аннотацию.
Обычно здесь отмечают, каких результатов удалось добиться, какие проблемы остались открытыми.

%%%% если имеется doi цитируемого источника, необходимо его указать, см. пример в \bibitem{article}
%%%% DOI публикации, зарегистрированной в системе Crossref, можно получить по адресу http://www.crossref.org/guestquery/
\begin{thebibliography}{99}
\bibitem{webArticle}
    \BibAuthor{Ray~Jiang, Silvia~Chiappa, Tor~Lattimore,Andr{\'a}s Gy{\"o}rgy, Pushmeet~Kohli}
    Degenerate Feedback Loops in Recommender Systems//
    \BibJournal{CoRR}, 2019, Vol. abs/1902.10730,
	  URL: \BibUrl{https://arxiv.org/abs/1902.10730}.

\bibitem{Article}
    \BibAuthor{Khritankov, Anton}
    Hidden Feedback Loops in Machine Learning Systems: A simulation Model and Preliminary Results//
    \BibJournal{Springer}, 2021, P.~54--65,

\bibitem{webArticle}
    \BibAuthor{Daniel Russo, Benjamin Van Roy, Abbas Kazerouni, Ian Osband}
    A Tutorial on Thompson Sampling//
    \BibJournal{CoRR}, 2017, Vol. abs/1707.02038,
	  URL: \BibUrl{https://arxiv.org/abs/1707.02038}.
\end{thebibliography}

%%%% если имеется doi цитируемого источника, необходимо его указать, см. пример в \bibitem{article}
%%%% DOI публикации, зарегистрированной в системе Crossref, можно получить по адресу http://www.crossref.org/guestquery/.

\end{document}
